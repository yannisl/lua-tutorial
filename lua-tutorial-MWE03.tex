\documentclass{tufte-book}
\usepackage[listings]{tcolorbox}
\lstloadlanguages{[LaTeX]TeX, [primitive]TeX,Pascal}
\usepackage{filecontents}
\usepackage{amsmath}
\usepackage{luacode} % loads luatexbase as well
\newcommand\emphasis[2][blue]{\lstset{emph={exec,if,then,else,do,end,while,for,print,sprint,directlua,#2},
   emphstyle={\ttfamily\textcolor{#1}}}}%
\lstset{language={[LaTeX]TeX},
      escapeinside={{(*@}{@*)}}, 
       gobble=0,
       stepnumber=1,numbersep=5pt, 
       numberstyle={\footnotesize\color{gray}},%firstnumber=last,
      breaklines=true,
      framesep=5pt,
      basicstyle=\small\ttfamily,
      showstringspaces=false,
      keywordstyle=\ttfamily\textcolor{blue},
      stringstyle=\color{orange},
      commentstyle=\color{black},
      rulecolor=\color{gray!10},
      breakatwhitespace=true,
      showspaces=false,  % shows spacing symbol
      backgroundcolor=\color{gray!15}}
\begin{document}

\emphasis{return,repeat,until,function,local}
\begin{tcblisting}{}
\begin{luacode}
-- example adapted from
-- http://rosettacode.org/wiki/Happy_numbers
function boxit(color, var, s)
  return "\\fbox{\\strut\\color{violet} "..var.."} "
end
function digits(n)
  if n > 0 then return n \% 10, digits(math.floor(n/10)) end
end
function sumsq(a, ...)
  return a and a ^ 2 + sumsq(...) or 0
end
local happy = setmetatable({true, false, false, false}, {
      __index = function(self, n)
         self[n] = self[sumsq(digits(n))]
         return self[n]
      end } )
i, j = 0, 8
repeat
   i, j = happy[j] and (tex.sprint(boxit(violet, j, " ")) or i+1) or i, j + 1
until i == 17 --or j > 999
\end{luacode}
\end{tcblisting}
\end{document}









