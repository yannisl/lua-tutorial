\documentclass{tufte-book}
\usepackage[listings]{tcolorbox}
\lstloadlanguages{[LaTeX]TeX, [primitive]TeX}
\usepackage{luacode} % loads luatexbase as well
% Emphasis
%\renewcommand{\ttdefault}{cmtt}			% prefer old tt font
\newcommand\emphasis[2][blue]{\lstset{emph={exec,if,then,else,do,end,while,for,print,sprint,directlua,#2},
   emphstyle={\ttfamily\textcolor{#1}}}}%

\lstset{language={[LaTeX]TeX},
      escapeinside={{(*@}{@*)}}, 
      numbers=left, 
      gobble=0,
      stepnumber=1,numbersep=5pt, 
       numberstyle={\footnotesize\color{gray}},%firstnumber=last,
      breaklines=true,
      framesep=5pt,
      basicstyle=\small\ttfamily,
      showstringspaces=false,
      keywordstyle=\ttfamily\textcolor{blue},
      stringstyle=\color{orange},
      commentstyle=\color{black},
      rulecolor=\color{gray!10},
      breakatwhitespace=true,
     showspaces=false,  % shows spacing symbol
	 xleftmargin=0pt,
	 xrightmargin=5pt,
	 aboveskip=3pt, % compact the code looks ugly in type
	 belowskip=7pt,  % user responsible to insert any skips
      backgroundcolor=\color{gray!15}}
\begin{document}

%\begin{tcblisting}{} uncomment if you have the latest version of tcolorbox
\begin{luacode}
  require "lfs"
  local temp=lfs.currentdir()
  tex.sprint(-2, temp)
  tex.sprint("\\par")
  function attrdir (path)
    for file in lfs.dir(path) do
        if file ~= "." and file ~= ".." then
            local f = path..'/'..file
            tex.sprint (-1, f.."\\par     ")
            local attr = lfs.attributes (f)
            assert (type(attr) == "table")
            if attr.mode == "directory" then
                attrdir (f)
            else
               -- for name, value in pairs(attr) do
                    --tex.sprint (-2,name, value)
                -- end
            end
        end
    end
  end
  local z="C:/test"
  attrdir (z)
\end{luacode}
%\end{tcblisting}
\end{document}









